% Options for packages loaded elsewhere
\PassOptionsToPackage{unicode}{hyperref}
\PassOptionsToPackage{hyphens}{url}
%
\documentclass[
]{article}
\usepackage{amsmath,amssymb}
\usepackage{lmodern}
\usepackage{iftex}
\ifPDFTeX
  \usepackage[T1]{fontenc}
  \usepackage[utf8]{inputenc}
  \usepackage{textcomp} % provide euro and other symbols
\else % if luatex or xetex
  \usepackage{unicode-math}
  \defaultfontfeatures{Scale=MatchLowercase}
  \defaultfontfeatures[\rmfamily]{Ligatures=TeX,Scale=1}
\fi
% Use upquote if available, for straight quotes in verbatim environments
\IfFileExists{upquote.sty}{\usepackage{upquote}}{}
\IfFileExists{microtype.sty}{% use microtype if available
  \usepackage[]{microtype}
  \UseMicrotypeSet[protrusion]{basicmath} % disable protrusion for tt fonts
}{}
\makeatletter
\@ifundefined{KOMAClassName}{% if non-KOMA class
  \IfFileExists{parskip.sty}{%
    \usepackage{parskip}
  }{% else
    \setlength{\parindent}{0pt}
    \setlength{\parskip}{6pt plus 2pt minus 1pt}}
}{% if KOMA class
  \KOMAoptions{parskip=half}}
\makeatother
\usepackage{xcolor}
\IfFileExists{xurl.sty}{\usepackage{xurl}}{} % add URL line breaks if available
\IfFileExists{bookmark.sty}{\usepackage{bookmark}}{\usepackage{hyperref}}
\hypersetup{
  hidelinks,
  pdfcreator={LaTeX via pandoc}}
\urlstyle{same} % disable monospaced font for URLs
\usepackage{color}
\usepackage{fancyvrb}
\newcommand{\VerbBar}{|}
\newcommand{\VERB}{\Verb[commandchars=\\\{\}]}
\DefineVerbatimEnvironment{Highlighting}{Verbatim}{commandchars=\\\{\}}
% Add ',fontsize=\small' for more characters per line
\newenvironment{Shaded}{}{}
\newcommand{\AlertTok}[1]{\textcolor[rgb]{1.00,0.00,0.00}{\textbf{#1}}}
\newcommand{\AnnotationTok}[1]{\textcolor[rgb]{0.38,0.63,0.69}{\textbf{\textit{#1}}}}
\newcommand{\AttributeTok}[1]{\textcolor[rgb]{0.49,0.56,0.16}{#1}}
\newcommand{\BaseNTok}[1]{\textcolor[rgb]{0.25,0.63,0.44}{#1}}
\newcommand{\BuiltInTok}[1]{#1}
\newcommand{\CharTok}[1]{\textcolor[rgb]{0.25,0.44,0.63}{#1}}
\newcommand{\CommentTok}[1]{\textcolor[rgb]{0.38,0.63,0.69}{\textit{#1}}}
\newcommand{\CommentVarTok}[1]{\textcolor[rgb]{0.38,0.63,0.69}{\textbf{\textit{#1}}}}
\newcommand{\ConstantTok}[1]{\textcolor[rgb]{0.53,0.00,0.00}{#1}}
\newcommand{\ControlFlowTok}[1]{\textcolor[rgb]{0.00,0.44,0.13}{\textbf{#1}}}
\newcommand{\DataTypeTok}[1]{\textcolor[rgb]{0.56,0.13,0.00}{#1}}
\newcommand{\DecValTok}[1]{\textcolor[rgb]{0.25,0.63,0.44}{#1}}
\newcommand{\DocumentationTok}[1]{\textcolor[rgb]{0.73,0.13,0.13}{\textit{#1}}}
\newcommand{\ErrorTok}[1]{\textcolor[rgb]{1.00,0.00,0.00}{\textbf{#1}}}
\newcommand{\ExtensionTok}[1]{#1}
\newcommand{\FloatTok}[1]{\textcolor[rgb]{0.25,0.63,0.44}{#1}}
\newcommand{\FunctionTok}[1]{\textcolor[rgb]{0.02,0.16,0.49}{#1}}
\newcommand{\ImportTok}[1]{#1}
\newcommand{\InformationTok}[1]{\textcolor[rgb]{0.38,0.63,0.69}{\textbf{\textit{#1}}}}
\newcommand{\KeywordTok}[1]{\textcolor[rgb]{0.00,0.44,0.13}{\textbf{#1}}}
\newcommand{\NormalTok}[1]{#1}
\newcommand{\OperatorTok}[1]{\textcolor[rgb]{0.40,0.40,0.40}{#1}}
\newcommand{\OtherTok}[1]{\textcolor[rgb]{0.00,0.44,0.13}{#1}}
\newcommand{\PreprocessorTok}[1]{\textcolor[rgb]{0.74,0.48,0.00}{#1}}
\newcommand{\RegionMarkerTok}[1]{#1}
\newcommand{\SpecialCharTok}[1]{\textcolor[rgb]{0.25,0.44,0.63}{#1}}
\newcommand{\SpecialStringTok}[1]{\textcolor[rgb]{0.73,0.40,0.53}{#1}}
\newcommand{\StringTok}[1]{\textcolor[rgb]{0.25,0.44,0.63}{#1}}
\newcommand{\VariableTok}[1]{\textcolor[rgb]{0.10,0.09,0.49}{#1}}
\newcommand{\VerbatimStringTok}[1]{\textcolor[rgb]{0.25,0.44,0.63}{#1}}
\newcommand{\WarningTok}[1]{\textcolor[rgb]{0.38,0.63,0.69}{\textbf{\textit{#1}}}}
\usepackage{longtable,booktabs,array}
\usepackage{calc} % for calculating minipage widths
% Correct order of tables after \paragraph or \subparagraph
\usepackage{etoolbox}
\makeatletter
\patchcmd\longtable{\par}{\if@noskipsec\mbox{}\fi\par}{}{}
\makeatother
% Allow footnotes in longtable head/foot
\IfFileExists{footnotehyper.sty}{\usepackage{footnotehyper}}{\usepackage{footnote}}
\makesavenoteenv{longtable}
\setlength{\emergencystretch}{3em} % prevent overfull lines
\providecommand{\tightlist}{%
  \setlength{\itemsep}{0pt}\setlength{\parskip}{0pt}}
\setcounter{secnumdepth}{-\maxdimen} % remove section numbering
\ifLuaTeX
  \usepackage{selnolig}  % disable illegal ligatures
\fi

\author{}
\date{}

\begin{document}

\hypertarget{ux5b9eux9a8cux62a5ux544a2}{%
\subsection{实验报告2}\label{ux5b9eux9a8cux62a5ux544a2}}

\hypertarget{ux59d3ux540dux5434ux6bc5ux9f99----ux5b66ux53f7pb19111749}{%
\subparagraph{姓名:吴毅龙
学号:PB19111749}\label{ux59d3ux540dux5434ux6bc5ux9f99----ux5b66ux53f7pb19111749}}

\hypertarget{1-ux95eeux9898}{%
\subsubsection{1. 问题}\label{1-ux95eeux9898}}

\begin{itemize}
\item
  分别编写\textbf{Newton迭代法和对分法}的程序,并利用它们去计算如下非线性方程的根\(f(x)=2^{-x}+e^x+2cosx-6\)

  其中,对分法的初始区间取\([0,10]\),Newton迭代法的初始点分别取0和10.
\item
  取误差限\(ε={10}^{-8}\),即当\(|f(x_k)|<ε\)时,停止迭代
\item
  列表给出每步的迭代结果或者前后5步的迭代结果(如果迭代步数超过10步时)以及迭代总步数;比较并分析两种方法的优劣。
\end{itemize}

\hypertarget{2-ux8ba1ux7b97ux8fc7ux7a0bux53caux8ba1ux7b97ux7ed3ux679c}{%
\subsubsection{2.
计算过程及计算结果}\label{2-ux8ba1ux7b97ux8fc7ux7a0bux53caux8ba1ux7b97ux7ed3ux679c}}

\hypertarget{21-ux7b97ux6cd5ux4e0eux7a0bux5e8fux5b9eux73b0}{%
\paragraph{2.1
算法与程序实现}\label{21-ux7b97ux6cd5ux4e0eux7a0bux5e8fux5b9eux73b0}}

\textbf{Newton迭代法}的基本流程为:确定迭代、变量建立迭代关系式、对迭代过程进行控制。

基于\textbf{二分法求方程的根}的原理是介值定理,即通过给定两个初始值a和b,使f(a)f(b)\textless0。如果f(x)是连续的,则必然在(a,b)内存在一个值c,使f(c)=0。

首先计算c=(a+b)/2;如果f(c)=0,则直接返回c即可,否则,进行下面的判断:如果f(a)f(c)\textgreater0,则令a=c,否则令b=c。

使用C语言对上述算法进行程序实现

\begin{Shaded}
\begin{Highlighting}[]
\PreprocessorTok{\#include}\ImportTok{\textless{}stdio.h\textgreater{}}
\PreprocessorTok{\#include}\ImportTok{\textless{}math.h\textgreater{}}
\PreprocessorTok{\#define e 2.718281828}
\PreprocessorTok{\#define epslion 1e{-}8}

\DataTypeTok{double}\NormalTok{ PrimitiveFunction}\OperatorTok{(}\DataTypeTok{double}\NormalTok{ x}\OperatorTok{)}
\OperatorTok{\{}
	\ControlFlowTok{return}\NormalTok{ pow}\OperatorTok{(}\DecValTok{2}\OperatorTok{,} \OperatorTok{{-}}\NormalTok{x}\OperatorTok{)} \OperatorTok{+}\NormalTok{ pow}\OperatorTok{(}\NormalTok{e}\OperatorTok{,}\NormalTok{ x}\OperatorTok{)} \OperatorTok{+} \DecValTok{2} \OperatorTok{*}\NormalTok{ cos}\OperatorTok{(}\NormalTok{x}\OperatorTok{)} \OperatorTok{{-}} \DecValTok{6}\OperatorTok{;}
\OperatorTok{\}}

\DataTypeTok{double}\NormalTok{ DerivativeFunction}\OperatorTok{(}\DataTypeTok{double}\NormalTok{ x}\OperatorTok{)}
\OperatorTok{\{}
	\ControlFlowTok{return} \OperatorTok{{-}}\DecValTok{1} \OperatorTok{*}\NormalTok{ pow}\OperatorTok{(}\DecValTok{2}\OperatorTok{,} \OperatorTok{{-}}\NormalTok{x}\OperatorTok{)} \OperatorTok{*}\NormalTok{ log}\OperatorTok{(}\DecValTok{2}\OperatorTok{)} \OperatorTok{+}\NormalTok{ pow}\OperatorTok{(}\NormalTok{e}\OperatorTok{,}\NormalTok{ x}\OperatorTok{)} \OperatorTok{{-}} \DecValTok{2} \OperatorTok{*}\NormalTok{ sin}\OperatorTok{(}\NormalTok{x}\OperatorTok{);}
\OperatorTok{\}}

\DataTypeTok{void}\NormalTok{ NewtonMethod}\OperatorTok{(}\DataTypeTok{double}\NormalTok{ x}\OperatorTok{)}
\OperatorTok{\{}
	\DataTypeTok{int}\NormalTok{ k }\OperatorTok{=} \DecValTok{0}\OperatorTok{;}
	\ControlFlowTok{while} \OperatorTok{(}\NormalTok{fabs}\OperatorTok{(}\NormalTok{PrimitiveFunction}\OperatorTok{(}\NormalTok{x}\OperatorTok{))} \OperatorTok{\textgreater{}=}\NormalTok{ epslion}\OperatorTok{)}
	\OperatorTok{\{}
\NormalTok{		printf}\OperatorTok{(}\StringTok{"k=\%d	x\%d=\%0.10f		f(x)=\%0.10f}\SpecialCharTok{\textbackslash{}n}\StringTok{"}\OperatorTok{,}\NormalTok{ k}\OperatorTok{,}\NormalTok{ k}\OperatorTok{,}\NormalTok{ x}\OperatorTok{,}\NormalTok{ PrimitiveFunction}\OperatorTok{(}\NormalTok{x}\OperatorTok{));}
\NormalTok{		x }\OperatorTok{=}\NormalTok{ x }\OperatorTok{{-}}\NormalTok{ PrimitiveFunction}\OperatorTok{(}\NormalTok{x}\OperatorTok{)} \OperatorTok{/}\NormalTok{ DerivativeFunction}\OperatorTok{(}\NormalTok{x}\OperatorTok{);}
\NormalTok{		k}\OperatorTok{++;}
	\OperatorTok{\}}
\NormalTok{	printf}\OperatorTok{(}\StringTok{"k=\%d	x\%d=\%0.10f		f(x)=\%0.10f}\SpecialCharTok{\textbackslash{}n}\StringTok{"}\OperatorTok{,}\NormalTok{ k}\OperatorTok{,}\NormalTok{ k}\OperatorTok{,}\NormalTok{ x}\OperatorTok{,}\NormalTok{ PrimitiveFunction}\OperatorTok{(}\NormalTok{x}\OperatorTok{));}
\OperatorTok{\}}

\DataTypeTok{double}\NormalTok{ Half}\OperatorTok{(}\DataTypeTok{double}\NormalTok{ x}\OperatorTok{,} \DataTypeTok{double}\NormalTok{ y}\OperatorTok{)}
\OperatorTok{\{}
\NormalTok{	x }\OperatorTok{=}\NormalTok{ x }\OperatorTok{*} \DecValTok{10000000}\OperatorTok{;}
\NormalTok{	y }\OperatorTok{=}\NormalTok{ y }\OperatorTok{*} \DecValTok{10000000}\OperatorTok{;}
	\ControlFlowTok{return} \OperatorTok{(}\NormalTok{x }\OperatorTok{+}\NormalTok{ y}\OperatorTok{)} \OperatorTok{/} \DecValTok{20000000}\OperatorTok{;}
\OperatorTok{\}}

\DataTypeTok{void}\NormalTok{ BisectionMethod}\OperatorTok{(}\DataTypeTok{double}\NormalTok{ low}\OperatorTok{,} \DataTypeTok{double}\NormalTok{ high}\OperatorTok{)}
\OperatorTok{\{}
	\DataTypeTok{int}\NormalTok{ k }\OperatorTok{=} \DecValTok{0}\OperatorTok{;}
	\DataTypeTok{double}\NormalTok{ root }\OperatorTok{=} \OperatorTok{(}\NormalTok{low }\OperatorTok{+}\NormalTok{ high}\OperatorTok{)} \OperatorTok{/} \DecValTok{2}\OperatorTok{;}
	\DataTypeTok{double}\NormalTok{ left }\OperatorTok{=}\NormalTok{ PrimitiveFunction}\OperatorTok{(}\NormalTok{low}\OperatorTok{);}
	\DataTypeTok{double}\NormalTok{ right }\OperatorTok{=}\NormalTok{ PrimitiveFunction}\OperatorTok{(}\NormalTok{high}\OperatorTok{);}
	\DataTypeTok{double}\NormalTok{ middle }\OperatorTok{=}\NormalTok{ PrimitiveFunction}\OperatorTok{(}\NormalTok{root}\OperatorTok{);}
	\ControlFlowTok{while} \OperatorTok{(}\NormalTok{fabs}\OperatorTok{(}\NormalTok{middle}\OperatorTok{)} \OperatorTok{\textgreater{}=}\NormalTok{ epslion}\OperatorTok{)}
	\OperatorTok{\{}
\NormalTok{		printf}\OperatorTok{(}\StringTok{"k=\%d	[\%0.10f,\%0.10f]		root=\%0.10f		f(x)=\%0.10f}\SpecialCharTok{\textbackslash{}n}\StringTok{"}\OperatorTok{,}\NormalTok{ k}\OperatorTok{,}\NormalTok{ low}\OperatorTok{,}\NormalTok{ high}\OperatorTok{,}\NormalTok{ root}\OperatorTok{,}\NormalTok{ middle}\OperatorTok{);}
		\ControlFlowTok{if} \OperatorTok{((}\NormalTok{middle }\OperatorTok{\textless{}}\NormalTok{epslion }\OperatorTok{\&\&}\NormalTok{ right }\OperatorTok{\textgreater{}}\NormalTok{ epslion}\OperatorTok{)||(}\NormalTok{middle }\OperatorTok{\textgreater{}}\NormalTok{ epslion }\OperatorTok{\&\&}\NormalTok{ right }\OperatorTok{\textless{}}\NormalTok{ epslion}\OperatorTok{))}
		\OperatorTok{\{}
\NormalTok{			low }\OperatorTok{=}\NormalTok{ root}\OperatorTok{;}
\NormalTok{			root }\OperatorTok{=}\NormalTok{ Half}\OperatorTok{(}\NormalTok{low}\OperatorTok{,}\NormalTok{ high}\OperatorTok{);}
\NormalTok{			left }\OperatorTok{=}\NormalTok{ PrimitiveFunction}\OperatorTok{(}\NormalTok{low}\OperatorTok{);}
\NormalTok{			middle }\OperatorTok{=}\NormalTok{ PrimitiveFunction}\OperatorTok{(}\NormalTok{root}\OperatorTok{);}
\NormalTok{			k}\OperatorTok{++;}
		\OperatorTok{\}}
		\ControlFlowTok{else}
		\OperatorTok{\{}
			\ControlFlowTok{if} \OperatorTok{((}\NormalTok{middle }\OperatorTok{\textless{}}\NormalTok{epslion }\OperatorTok{\&\&}\NormalTok{ left }\OperatorTok{\textgreater{}}\NormalTok{ epslion}\OperatorTok{)} \OperatorTok{||} \OperatorTok{(}\NormalTok{middle }\OperatorTok{\textgreater{}}\NormalTok{ epslion}\OperatorTok{\&\&}\NormalTok{ left }\OperatorTok{\textless{}}\NormalTok{ epslion}\OperatorTok{))}
			\OperatorTok{\{}
\NormalTok{				high }\OperatorTok{=}\NormalTok{ root}\OperatorTok{;}
\NormalTok{				root }\OperatorTok{=}\NormalTok{ Half}\OperatorTok{(}\NormalTok{low}\OperatorTok{,}\NormalTok{ high}\OperatorTok{);}
\NormalTok{				right }\OperatorTok{=}\NormalTok{ PrimitiveFunction}\OperatorTok{(}\NormalTok{high}\OperatorTok{);}
\NormalTok{				middle }\OperatorTok{=}\NormalTok{ PrimitiveFunction}\OperatorTok{(}\NormalTok{root}\OperatorTok{);}
\NormalTok{				k}\OperatorTok{++;}
			\OperatorTok{\}}
			\ControlFlowTok{else}
			\OperatorTok{\{}
\NormalTok{				printf}\OperatorTok{(}\StringTok{"ERROR!}\SpecialCharTok{\textbackslash{}n}\StringTok{"}\OperatorTok{);}
				\ControlFlowTok{break}\OperatorTok{;}
			\OperatorTok{\}}
		\OperatorTok{\}}
	\OperatorTok{\}}
\NormalTok{	printf}\OperatorTok{(}\StringTok{"k=\%d	[\%0.10f,\%0.10f]		root=\%0.10f		f(x)=\%0.10f}\SpecialCharTok{\textbackslash{}n}\StringTok{"}\OperatorTok{,}\NormalTok{ k}\OperatorTok{,}\NormalTok{ low}\OperatorTok{,}\NormalTok{ high}\OperatorTok{,}\NormalTok{ root}\OperatorTok{,}\NormalTok{ middle}\OperatorTok{);}
\OperatorTok{\}}

\DataTypeTok{int}\NormalTok{ main}\OperatorTok{()}
\OperatorTok{\{}
	\DataTypeTok{double}\NormalTok{ x1}\OperatorTok{,}\NormalTok{ x2}\OperatorTok{;}
	\DataTypeTok{double}\NormalTok{ low}\OperatorTok{,}\NormalTok{ high}\OperatorTok{;}
\NormalTok{	scanf}\OperatorTok{(}\StringTok{"\%lf \%lf"}\OperatorTok{,} \OperatorTok{\&}\NormalTok{x1}\OperatorTok{,} \OperatorTok{\&}\NormalTok{x2}\OperatorTok{);}
\NormalTok{	scanf}\OperatorTok{(}\StringTok{"\%lf \%lf"}\OperatorTok{,} \OperatorTok{\&}\NormalTok{low}\OperatorTok{,} \OperatorTok{\&}\NormalTok{high}\OperatorTok{);}

\NormalTok{	printf}\OperatorTok{(}\StringTok{"Newton Method for x1}\SpecialCharTok{\textbackslash{}n}\StringTok{"}\OperatorTok{);}
\NormalTok{	NewtonMethod}\OperatorTok{(}\NormalTok{x1}\OperatorTok{);}
\NormalTok{	printf}\OperatorTok{(}\StringTok{"}\SpecialCharTok{\textbackslash{}n}\StringTok{"}\OperatorTok{);}

\NormalTok{	printf}\OperatorTok{(}\StringTok{"Newton Method for x2}\SpecialCharTok{\textbackslash{}n}\StringTok{"}\OperatorTok{);}
\NormalTok{	NewtonMethod}\OperatorTok{(}\NormalTok{x2}\OperatorTok{);}
\NormalTok{	printf}\OperatorTok{(}\StringTok{"}\SpecialCharTok{\textbackslash{}n}\StringTok{"}\OperatorTok{);}


\NormalTok{	printf}\OperatorTok{(}\StringTok{"Bisection Method}\SpecialCharTok{\textbackslash{}n}\StringTok{"}\OperatorTok{);}
\NormalTok{	BisectionMethod}\OperatorTok{(}\NormalTok{low}\OperatorTok{,}\NormalTok{ high}\OperatorTok{);}
	\ControlFlowTok{return} \DecValTok{0}\OperatorTok{;}
\OperatorTok{\}}
\end{Highlighting}
\end{Shaded}

\hypertarget{22-ux8ba1ux7b97ux7ed3ux679c}{%
\paragraph{2.2 计算结果}\label{22-ux8ba1ux7b97ux7ed3ux679c}}

\hypertarget{221-newtonux8fedux4ee3ux6cd5}{%
\subparagraph{2.2.1 Newton迭代法}\label{221-newtonux8fedux4ee3ux6cd5}}

\begin{itemize}
\item
  初值\(x_0=0\)的Newton迭代结果
\end{itemize}

\begin{longtable}[]{@{}lll@{}}
\toprule
迭代步数k & \(x_k\) & \(f(x_k)\) \\
\midrule
\endhead
\(k=0\) & 0.0000000000 & -2.0000000000 \\
\(k=1\) & 6.5177827065 & 673.0315732306 \\
\(k=2\) & 5.5230611395 & 245.8716230513 \\
\(k=3\) & 4.5464628865 & 88.0107359569 \\
\(k=4\) & 3.6319814821 & 30.1039795420 \\
\(k=5\) & 2.8535707847 & 9.5703611468 \\
\(k=6\) & 2.2800010916 & 2.6801276990 \\
\(k=7\) & 1.9497821270 & 0.5460548492 \\
\(k=8\) & 1.8403446404 & 0.0453715661 \\
\(k=9\) & 1.8294833422 & 0.0004091171 \\
\(k=10\) & 1.8293836107 & 0.0000000342 \\
\(k=11\) & 1.8293836024 & 0.0000000000 \\
\bottomrule
\end{longtable}

\begin{itemize}
\item
  初值\(x_0=10\)的Newton迭代结果
\end{itemize}

\begin{longtable}[]{@{}lll@{}}
\toprule
迭代步数k & \(x_k\) & \(f(x_k)\) \\
\midrule
\endhead
\(k=0\) & 10.0000000000 & 22018.7885911142 \\
\(k=1\) & 9.0003978895 & 8098.4880529237 \\
\(k=2\) & 8.0012609805 & 2978.4296918299 \\
\(k=3\) & 7.0027054717 & 1095.1161260613 \\
\(k=4\) & 6.0055867983 & 401.6279733817 \\
\(k=5\) & 5.0169063711 & 145.5741686140 \\
\(k=6\) & 4.0643840546 & 51.0816097141 \\
\(k=7\) & 3.2099219086 & 16.8898913273 \\
\(k=8\) & 2.5299415504 & 5.0885160278 \\
\(k=9\) & 2.0790045507 & 1.2599570918 \\
\(k=10\) & 1.8719524207 & 0.1809311653 \\
\(k=11\) & 1.8308468174 & 0.0060087418 \\
\(k=12\) & 1.8293853949 & 0.0000073519 \\
\(k=13\) & 1.8293836024 & 0.0000000000 \\
\bottomrule
\end{longtable}

\hypertarget{222-ux5bf9ux5206ux6cd5}{%
\subparagraph{2.2.2 对分法}\label{222-ux5bf9ux5206ux6cd5}}

\begin{longtable}[]{@{}llll@{}}
\toprule
迭代步数k & 区间 & 区间中点 & \(f(x_k)\) \\
\midrule
\endhead
\(k=0\) & {[}0.0000000000,10.0000000000{]} & 5.0000000000 &
143.0117333482 \\
\(k=1\) & {[}0.0000000000,5.0000000000{]} & 2.5000000000 &
4.7569834198 \\
\(k=2\) & {[}0.0000000000,2.5000000000{]} & 1.2500000000 &
-1.4585641109 \\
\(k=3\) & {[}1.2500000000,2.5000000000{]} & 1.8750000000 &
0.1943790391 \\
\(k=4\) & {[}1.2500000000,1.8750000000{]} & 1.5625000000 &
-0.8741104693 \\
\(k=5\) & {[}1.5625000000,1.8750000000{]} & 1.7187500000 &
-0.4134649413 \\
.......... & .......... & .......... & .......... \\
\(k=25\) & {[}1.8293833733,1.8293836713{]} & 1.8293835223 &
-0.0000003287 \\
\(k=26\) & {[}1.8293835223,1.8293836713{]} & 1.8293835968 &
-0.0000000231 \\
\(k=27\) & {[}1.8293835968,1.8293836713{]} & 1.8293836340 &
0.0000001297 \\
\(k=28\) & {[}1.8293835968,1.8293836340{]} & 1.8293836154 &
0.0000000533 \\
\(k=29\) & {[}1.8293835968,1.8293836154{]} & 1.8293836061 &
0.0000000151 \\
\(k=30\) & {[}1.8293835968,1.8293836061{]} & 1.8293836014 &
-0.0000000040 \\
\bottomrule
\end{longtable}

\hypertarget{3-ux8ba1ux7b97ux7ed3ux679cux5206ux6790}{%
\subsubsection{3.
计算结果分析}\label{3-ux8ba1ux7b97ux7ed3ux679cux5206ux6790}}

从上述实验结果可以看出,对于不同的起始数据,牛顿迭代法的收敛速度是不一样的。越靠近根的起始数据的收敛速度越快。横向对比两个求根算法,可以看出牛顿迭代法的收敛速度比对分法快很多,牛顿迭代法只需要十几次就可以得出目标精度的结果,而使用对分法却需要进行多大三十次的计算。

总体来看,虽然少数情况下牛顿迭代法不能收敛,但是大多数情况下它效果都非常好。对分法固定每次缩短一半的区间,而牛顿迭代法的迭代效率往往更高,一般情况下使用牛顿迭代法可以获得更快的收敛速度。

\hypertarget{4-ux5b9eux9a8cux603bux7ed3}{%
\subsubsection{4. 实验总结}\label{4-ux5b9eux9a8cux603bux7ed3}}

通过这个实验,我再一次体会到算法从理论落地到实践的过程,这其中需要解决许多在理论推演过程中无法想象到的问题。比如在实验过程中,我遇到最大的问题就是double数据类型精度的损失问题。在对分法求根的程序编写过程中,我最初使用的是,区间两端的函数值相乘是否小于0作为下一次计算区间选择的判据。但是在程序调试中发现,这样的判断依据会使程序陷入无限循环。究其原因就是double双精度数据在相乘的过程中精度发生了损失。因此我针对性的修改了判据,从而解决了程序的问题。

\end{document}
